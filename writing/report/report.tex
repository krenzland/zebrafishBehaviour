\documentclass[nobib]{tufte-handout}

\title{Data-Driven Models for Zebrafish Motion}

\author[Lukas Krenz]{Lukas Krenz}

%\geometry{showframe} % display margins for debugging page layout
\usepackage{hyphenat}
\usepackage[
  style=verbose,
  autocite=footnote,
  backend=biber
]{biblatex}
\addbibresource{../bibliography.bib}

\usepackage{caption}
\usepackage{xpatch}
\usepackage{bm}
\usepackage{amsmath}
\usepackage{mathtools} % for \mathclap
\usepackage{varioref}
\usepackage{siunitx}
\usepackage{hyperref}
\usepackage[noabbrev]{cleveref}
\newcommand{\creflastconjunction}{, and\nobreakspace} % use Oxford comma
\usepackage{todonotes}
\usepackage{phaistos}
\usepackage{multimedia}
\usepackage{tikz}
\usetikzlibrary{arrows, positioning, shapes.geometric}
\usetikzlibrary{calc}

\graphicspath{{../../figures/}}

% \usepackage{graphicx} % allow embedded images
%   \setkeys{Gin}{width=\linewidth,totalheight=\textheight,keepaspectratio}
%   \graphicspath{{graphics/}} % set of paths to search for images
% \usepackage{amsmath}  % extended mathematics
% \usepackage{booktabs} % book-quality tables
% \usepackage{units}    % non-stacked fractions and better unit spacing
% \usepackage{multicol} % multiple column layout facilities
% \usepackage{fancyvrb} % extended verbatim environments
%   \fvset{fontsize=\normalsize}% default font size for fancy-verbatim environments

% Standardize command font styles and environments
\newcommand{\doccmd}[1]{\texttt{\textbackslash#1}}% command name -- adds backslash automatically
\newcommand{\docopt}[1]{\ensuremath{\langle}\textrm{\textit{#1}}\ensuremath{\rangle}}% optional command argument
\newcommand{\docarg}[1]{\textrm{\textit{#1}}}% (required) command argument
\newcommand{\docenv}[1]{\textsf{#1}}% environment name
\newcommand{\docpkg}[1]{\texttt{#1}}% package name
\newcommand{\doccls}[1]{\texttt{#1}}% document class name
\newcommand{\docclsopt}[1]{\texttt{#1}}% document class option name
\newenvironment{docspec}{\begin{quote}\noindent}{\end{quote}}% command specification environment

\begin{document}

\maketitle% this prints the handout title, author, and date

\begin{abstract}
\noindent
The goal of the project is to compare different data-driven models for the behavior of zebrafish.
The models should be able to predict and simulate the individual motion of an animal reacting to its environment (e.g. another fish, wall).
It can be used to steer a fish in a virtual reality environment, for example.

We start with the evaluation of a simple force based model.
The model consists of two parts, one that models the information processing and one that models the gliding phase.
The gliding is described by a physical model with parameters taken from the data.

We will then enhance the model by using a spatio-temporal (linear) receptive field as the social model that includes information about the perception of the fish.
This modeling strategy is inspired by computational neuroscience and adds a 'memory' to the model.

As a next step, the receptive field will be replaced by a recurrent neural network.
This is a generalization of the previous model and includes possible non-linear reactions and time-dependencies.
\end{abstract}

\section{Introduction}
The goal of the project is to compare different strategies for modelling the behaviour of juvenile zebrafish.
These models should be able to predict and simulate the individual motion of one animal reacting to its environment (e.g. another fish and a wall).

A use-case for this project is steering a virtual fish in a virtual reality environment for animals.
It can be used to perform experiments that investigate causal relationships in animal behaviour.

The movement of zebrafish can be described by discrete models that assume that the fish moves in a piece-wise linear fashion.
After choosing a heading direction, the fish kicks off and moves in an approximately straight line.
We model the heading change for each kick.

We start with an simple model and refine it twice.
Each modification drops assumptions about the behaviour and thus creates a more flexible, data-driven model.
The models developed in the earlier steps serve as baselines for the more complicated ones.
We can thus see whether the assumptions are correct and how important each component of the model is.
\begin{enumerate}
\item We start from the simple model published by Calovi et.\ al.\footnote{: Calovi et al (2017). Disentangling and modelling interactions in fish with burst-and-coast swimming}.
It is capable of simulating one rummy-nose tetra moving in a tank, including its interaction with another fish and the boundary enclosing the arena.

The model consists of two parts.
The first part models the social information processing, that is, how the fish chooses its heading direction with respect to the environmental forces.
The second part models the gliding after a kick and assumes that the fish moves in a straight line with decaying velocity.
All parameters of the model are obtained directly from the data.

We need to adapt some details of this model to the movement of zebrafish.
This is possible because rummy-nose tetra and juvenile zebrafish move in a similar way.
The model makes the assumption that the direction of each kick is influenced only by the current environment of the fish.
\item The next model drops this assumption.
We will replace the social model by a linear spatio-temporal receptive field model.
This model considers the movement of other fish over a certain time period.
Practically speaking, this adds a memory to the model.

We hope that this will result in a model that reflects the biological reality more closely.
\item As a next step, the simple receptive field model will be replaced by a more general one, using a neural network.
This type of model still has a memory but additionally is able to approximate a larger family of functions.
Due to the strong non-linearity this model is hard to interpret but able to capture complex, possible non-linear behaviour.
\end{enumerate}

\section{Models}
We begin with the modeling of wall-forces.
We follow the modeling approach of~\autocite{calovi} and describe the influence of a wall on the heading change \(\delta \phi_w\) as
\begin{equation*}
  \delta \phi_w (r_w, \theta_w) = f(r_w)O_w(\theta_w),
\end{equation*}
where $r_w$ corresponds to the distance to the wall and $\theta$ is the relative angle of the fish towars the wall.
We split the wall influence into an exponential decaying force term \(f_w\) and an odd angular-response function \(O_w\).
%todo: multiply by a constant
\begin{align*}
  f(r_w) &= \exp\left( -{(r_w/l_w)}^2 \right), \\
  O(\theta_w) &= \left(a_1 \sin(\theta_w) + a_2 \sin(2  \theta_w)  \right)  \left(1 +  b_1  \cos(\theta_w) + b_2 \cos(2  \theta_w) \right),
\end{align*}
where $l_w, a_1, a_2, b_1 \text{ and } b_2$ are parameters.
Note that we use a different series expansion for the odd angular function \(O_w\)
\footnote{Their proposed form does not work in our case.
  The reason for this could be that they consider both a different species and a round wall.}.
Finally, we consider the closest two walls and sum over the influences
\begin{equation*}
 \delta \phi_w^{\text{total}} \left( \bm{r_w}, \bm{\theta_w} \right) = \sum_{i \in 2 \text{ closest walls}} \delta \phi_w^i (r_w^i, \theta_w^i).
\end{equation*}
The parameters are fit by minimizing the mean-squared error of heading change prediction using gradient-descent.
% todo: we do not actually use gradient descent but something different.

\subsection{Receptive Fields}
We now explain the input features of our social models.
From here on, we restrict our dataset to kicks where the wall influence is neglible.
Practically, we drop all kicks where the value of \(f_w(r_w)\) is larger than $0.1$.

Our social model use a receptive field as their input~\autocite{discreteModes}.
We construct this by rotating the coordinate system such that the fish we are considering is parallel to the x-axis.
This corresponds to a heading of zero degrees.
We then shift the coordinate system such that our fish is at the center.

Or models then try to predict the product of a vector in direction of the heading change and the kick length.
This correspond to the kick trajectory in our new coordinate system.

As features we use a discretized position and the angle of the other fish.
The position is discretized using adaptive spatial bins that are constructed by\(\cdots\).


One hot encoidng, alternatively encode x and y seperately, unfair to linear model.

\subsection{First linear model}

\subsection{Adding a time dependence}

\subsection{Adding non-linear effects}

\section{Implementation \textit{\&} Training Details}

\section{Evaluation}

\section{Conclusion}

\end{document}