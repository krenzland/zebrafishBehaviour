% Created 2018-02-01 Thu 22:30
\documentclass[11pt]{article}
\usepackage[utf8]{inputenc}
\usepackage[T1]{fontenc}
\usepackage{fixltx2e}
\usepackage{graphicx}
\usepackage{longtable}
\usepackage{float}
\usepackage{wrapfig}
\usepackage{rotating}
\usepackage[normalem]{ulem}
\usepackage{amsmath}
\usepackage{textcomp}
\usepackage{marvosym}
\usepackage{wasysym}
\usepackage{amssymb}
\usepackage{hyperref}
\tolerance=1000
\usepackage{pgfgantt}
\usepackage{microtype}
\author{Lukas Krenz}
\date{\today}
\title{Project description: Data-Driven Models for Zebrafish Motion}
\hypersetup{
  pdfkeywords={},
  pdfsubject={},
  pdfcreator={Emacs 25.3.1 (Org mode 8.2.10)}}
\begin{document}

\maketitle

This project is done in collaboration with the Couzin Lab at the Max Plank Institute for Ornithology and the University of Constance.
The goal of the project is to compare different strategies for modelling the behaviour of juvenile zebrafish.
These models should be able to predict and simulate the individual motion of one animal reacting to its environment (e.g. another fish and a wall).

A use-case for this project is steering a virtual fish in a virtual reality environment for animals.
It can be used to perform experiments that investigate causal relationships in animal behaviour.

The movement of zebrafish can be described by discrete models that assume that the fish moves in a piece-wise linear fashion.
After choosing a heading direction, the fish kicks off and moves in an approximately straight line.
We model the heading change for each kick.

We start with an simple model and refine it twice.
Each modification drops assumptions about the behaviour and thus creates a more flexible, data-driven model.
The models developed in the earlier steps serve as baselines for the more complicated ones.
We can thus see whether the assumptions are correct and how important each component of the model is.
\begin{enumerate}
\item We start from the simple model published by Calovi et.\ al.\footnote{: Calovi et al (2017). Disentangling and modelling interactions in fish with burst-and-coast swimming}.
It is capable of simulating one rummy-nose tetra moving in a tank, including its interaction with another fish and the boundary enclosing the arena.

The model consists of two parts.
The first part models the social information processing, that is, how the fish chooses its heading direction with respect to the environmental forces.
The second part models the gliding after a kick and assumes that the fish moves in a straight line with decaying velocity.
All parameters of the model are obtained directly from the data.

We need to adapt some details of this model to the movement of zebrafish.
This is possible because rummy-nose tetra and juvenile zebrafish move in a similar way.
The model makes the assumption that the direction of each kick is influenced only by the current environment of the fish.
\item The next model drops this assumption.
We will replace the social model by a linear spatio-temporal receptive field model.
This model considers the movement of other fish over a certain time period.
Practically speaking, this adds a memory to the model.

We hope that this will result in a model that reflects the biological reality more closely.
\item As a next step, the simple receptive field model will be replaced by a more general one, using a neural network.
This type of model still has a memory but additionally is able to approximate a larger family of functions.
Due to the strong non-linearity this model is hard to interpret but able to capture complex, possible non-linear behaviour.
\end{enumerate}

The project is accompanied by the following two lectures:
\begin{description}
\item[Topics in Computational Biology] 
This lecture describes how mathematical models can be applied to other areas of biology such as cell biology.
It discusses models stemming from vastly different areas of applied mathematics such as dynamical systems and probabilistic modelling. 
\item[Computational Neuroscience---A Lecture Series from Models to Applications]
Both receptive fields and neural networks are inspired by models from computational neuroscience.
Basic knowledge of this subject area provides important background information, including but not limited to terminology and an understanding of the biological motivation for those models.
Additionally, the lecture connects fundamental neuroscience research with medical engineering topics.
This leads to an introduction to computational neuroscience that is focused on practical methodology applied in modern medicine.
\end{description}
Both lectures are concerned with small-scale models and their practical application and serve as a supplement to the coarse-grained models explored in this project.

The timeline and milestones of the project are:
\begin{description}
\item[Pre-Processing]
We start with the pre-processing of the kick data.
\item[Classical Models: Calovi \textit{\&} Receptive Field]
The next step is then the implementation of the classical models, i.e.\ the Calovi and receptive field model.
\item[Neural Network]
Finally we implement a neural network based model.
\end{description}
Throughout the project we continuously analyse the biological plausibility of our models and evaluate their effectiveness.

The expected time for each milestone is as follows:
\begin{itemize}
\item Pre-processing: 2 months
\item Calovi-Model: 2 months
\item Spatio-Temporal Receptive Field: 1 month
\item Neural Network: 1 month
\end{itemize}
% Emacs 25.3.1 (Org mode 8.2.10)
\end{document}